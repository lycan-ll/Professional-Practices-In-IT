\documentclass{article}
\usepackage{enumitem}
\usepackage{geometry}
\usepackage{xcolor}
\usepackage{titling}
\usepackage{titlesec}

% Set page margins
\geometry{margin=1in}

% Define colors
\definecolor{myblue}{RGB}{0, 70, 129}

% Customize section and subsection titles
\titleformat{\section}[block]{\color{myblue}\Large\bfseries}{\thesection}{1em}{}
\titleformat{\subsection}[block]{\color{myblue}\large\bfseries}{\thesubsection}{1em}{}

% Customize itemize environment
\setlist[itemize,1]{label=--,left=0.5in}
\setlist[itemize,2]{label=--,left=0.5in}

% Set title information
\renewcommand{\maketitle}{
  \begin{center}
    \huge\bfseries\thetitle
  \end{center}
}


\title{Izaan Mohtashim \\ p20-0613\\"A Tale of Two Democracies: USA and Pakistan Elections"}


\begin{document}

\maketitle

In the United States, the electoral process unfolds through a series of distinct stages. Initially, political parties select their nominees via primary elections or caucuses. Subsequently, large-scale gatherings known as national conventions officially endorse the candidates for the positions of President and Vice President.

When Americans cast their votes in the Presidential election, they are technically voting for a body called the Electoral College rather than directly for the President. The candidate who secures the majority of Electoral College votes ultimately assumes the presidency.

Congressional elections occur every two years, determining the composition of the House of Representatives and a portion of the Senate. Concurrently, states conduct elections for various positions, including that of governor.

The option to register as a voter is available, with some regions allowing early voting or mail-in ballots for those unable to vote in person. Following the counting of votes, winners are declared and subsequently assume their respective offices. The overarching objective is to ensure that the government accurately reflects the choices made by the electorate.

\section{Key Differences in Electoral Systems}

\subsection{Electoral College vs. Direct Presidential Vote}
\begin{itemize}
    \item In the United States, the President is indirectly elected through the Electoral College, composed of electors chosen by individual states. This system can result in instances where the popular vote winner does not secure the presidency.
    \item In Pakistan, the President is directly elected through a popular vote, wherein citizens cast their ballots for their preferred candidate.
\end{itemize}

\subsection{Bicameral vs. Unicameral Legislature}
\begin{itemize}
    \item The United States boasts a bicameral legislature, comprising the Senate and the House of Representatives, with Senators serving six-year terms and House members serving two-year terms.
    \item Pakistan's National Assembly operates as a unicameral body, consisting of a single chamber, and its members serve five-year terms.
\end{itemize}

\subsection{Parliamentary vs. Presidential System}
\begin{itemize}
    \item The United States adheres to a presidential system, characterized by separate elections for the President and the Congress (House and Senate), each possessing distinct powers.
    \item Pakistan follows a parliamentary system, where the Prime Minister typically leads the majority party in the National Assembly, and executive authority is concentrated within the legislature.
\end{itemize}

\subsection{Proportional Representation vs. First-Past-the-Post}
\begin{itemize}
    \item Pakistan employs a mixed electoral system, with some seats allocated through proportional representation (PR) and others through a first-past-the-post (FPTP) system, allowing for increased representation of smaller political parties.
    \item The United States primarily employs a first-past-the-post system in Congressional elections, where the candidate with the most votes in a district emerges victorious.
\end{itemize}

\subsection{Role of Political Parties}
\begin{itemize}
    \item The United States is dominated by two major political parties, the Democrats and the Republicans, both exerting substantial influence on electoral outcomes.
    \item In Pakistan, a multitude of political parties operate with varying degrees of impact, leading to a more diverse political landscape.
\end{itemize}

\subsection{Voter Registration and Identification}
\begin{itemize}
    \item Voter registration procedures in the United States vary by state, with some jurisdictions requiring voter identification.
    \item Pakistan has established the National Database and Registration Authority (NADRA), responsible for maintaining voter records and issuing computerized national identity cards (CNICs), often serving as voter identification.
\end{itemize}

\subsection{Military Intervention}
\begin{itemize}
    \item Pakistan has witnessed periods of military involvement in politics, resulting in interruptions to civilian governance.
    \item Conversely, the United States has not experienced a history of military intervention in electoral processes or government affairs.
\end{itemize}

These disparities are rooted in each country's unique political systems, electoral regulations, and historical backgrounds, significantly influencing the conduct of elections and the selection of leaders.

These differences arise from distinct methodologies and historical trajectories, reflecting each country's distinct approaches and systems.

\end{document}

