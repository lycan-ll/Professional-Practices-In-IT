\documentclass{article}
\usepackage[a4paper, margin=2.5cm]{geometry}
\usepackage{fancyhdr}
\usepackage{lipsum}
\usepackage{graphicx}

% Define the page style with a header and footer
\pagestyle{fancy}
\fancyhf{}
\renewcommand{\headrulewidth}{0pt}

\title{Enron Corporation - Ethical Scandal}
\author{Izaan Mohtashim \\ 20P-0613 \\ SEC:7-A}
\date{\today}



\begin{document}

\maketitle
\begin{figure}[h]
    \centering
    \includegraphics[width=0.5\textwidth]{enron.png}
    \caption{Enron}
\end{figure}

\section*{Case Description:}

\textit{Enron Corporation}, once a leading energy company, faced a major ethical scandal related to accounting fraud and corporate governance. The company engaged in deceptive accounting practices, hiding debt and inflating profits through complex financial structures and partnerships.

\subsection*{Stakeholder Network:}
\begin{enumerate}
    \item \textbf{Employees and Retirees:} Thousands of employees lost their jobs, and many lost their retirement savings due to the collapse of Enron.
    \item \textbf{Shareholders and Investors:} Shareholders suffered massive financial losses as the stock price plummeted.
    \item \textbf{Regulatory Bodies and Government Agencies:} Investigated the company for accounting fraud, leading to regulatory reforms like the Sarbanes-Oxley Act.
    \item \textbf{Creditors and Business Partners:} Faced financial losses due to Enron's bankruptcy and questionable business practices.
    \item \textbf{Public and Media:} The scandal damaged public trust in corporate integrity and led to scrutiny of corporate governance practices.
\end{enumerate}

\subsection*{Ethical Issues:}
\begin{enumerate}
    \item \textbf{Accounting Fraud:} Enron engaged in misleading accounting practices to portray a healthier financial picture than was accurate, deceiving investors and stakeholders.
    \item \textbf{Corporate Governance:} Weak corporate governance structures allowed for conflicts of interest and lack of oversight, contributing to the fraudulent activities.
    \item \textbf{Employee Impact:} Employees faced job losses and financial ruin, highlighting the consequences of corporate mismanagement on workers.
\end{enumerate}

\section*{Course of Actions:}
\begin{enumerate}
    \item \textbf{Bankruptcy and Legal Proceedings:} Enron declared bankruptcy in 2001, leading to numerous legal proceedings, including criminal trials of top executives.
    \item \textbf{Regulatory Reforms:} The scandal prompted regulatory reforms aimed at improving transparency and corporate governance, such as the Sarbanes-Oxley Act in the United States.
    \item \textbf{Restitution and Settlements:} Enron's creditors and investors received settlements from lawsuits against the company's executives and auditors.
    \item \textbf{Repercussions for Executives:} Several top Enron executives were convicted of fraud, conspiracy, and insider trading, facing prison sentences.
\end{enumerate}

\section*{Evaluation:}
The Enron scandal is a stark reminder of the dangers of corporate fraud, inadequate oversight, and unethical practices within a company. The collapse of Enron had far-reaching implications, leading to significant regulatory changes and reforms in corporate governance. The case highlighted the importance of transparency, accountability, and ethical conduct in corporate culture and governance to protect stakeholders' interests and maintain public trust.

\end{document}